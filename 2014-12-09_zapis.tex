\documentclass[11pt,a4paper]{article}

% ===== LOADING PACKAGES =====
% language settings, main documnet language last
\usepackage[czech]{babel}
% enabling new fonts support (nicer)
\usepackage{lmodern}
% setting input encoding
\usepackage[utf8]{inputenc}
% setting output encoding
\usepackage[T1]{fontenc}
% set page margins
\usepackage[top=2cm, bottom=2cm, left=2cm, right=2cm]{geometry}
% package to make bullet list nicer
\usepackage{enumitem}

\begin{document}
\section*{Zápis zo zakladacej schôdze Spolku přátel severské zvěře}
\textbf{Dátum:} 9.\ 12.\ 2014\\
\textbf{Prítomní:} Jan Drábek, Jan Mrázek, Martin Ukrop, Vladimír Štill, Martina Krasnayová, Martin Hanžl, Ondřej Slámečka, Henrich Lauko, Karel Kubíček, František Blahoudek, Tomáš Effenberger, Jiří Novotný, Jiří Marek, Jiří Mauritz, Matej Kollár, Gabriela Véghová, Jaroslav Čechák

\subsection*{Stanovy spolku a prijatie členov}
\begin{itemize}[itemsep=0pt]
\item Budúcim členom boli predstavené motivácie k založeniu spolku a základné body stanov. Prítomní nemali žiadne výhrady k aktuálnemu zneniu stanov.
\item Pri založení spolku sa aktívnimi členmi stali: Jan Drábek, Jan Mrázek, Martin Ukrop, Vladimír Štill, Martina Krasnayová, Martin Hanžl, Ondřej Slámečka, Henrich Lauko, Karel Kubíček, František Blahoudek, Tomáš Effenberger, Jiří Novotný a Jiří Marek.
\item Pri založení spolku sa sympatizujúcimi členmi stali: Jiří Mauritz, Matej Kollár, Gabriela Véghová a Jaroslav Čechák.
\item Prítomní aktívni členovia jednohlasne zvolili výbor spolku (podľa zloženia prípravného výboru): Martin Ukrop, Jan Drábek, Martina Krasnayová, Martin Hanžl a Jan Mrázek.
\item Prítomní aktívni členovia jednohlasne zvolili Martina Ukropa predsedom spolku.
\end{itemize}

\subsection*{Plánované akcie spolku}
\begin{itemize}[itemsep=0pt]
\item \textbf{FIORD} -- náborová akcia testerov/B-týmákov InterSoBa a l\`{a} Moravské hvozdy. Súboj s papierovými guľami na gumičke a štítmi, ideálne pred FI alebo v átriu. Promo akcia na prednáške Petra Švendu (Úvod do jazyka C). Predpokladaný termín je niekedy začiatkom jarného semestra.
\item \textbf{FI-complex} -- snaha o uskutočnenie zrušenej šiforvacej hry v priestoroch fakulty. Termín ešte nejasný.
\item \textbf{Schrödingerov grill} -- opätovné usporiadanie grilovačny pre študentov a zamestnancov FI v átriu. Poznámka: Bude potrebné zistiť, nakoľko podobné akcie obmedzujú požiarne a hygienické predpisy. Plánovaný termín: keď bude teplo, asi niekedy koncom jarného semestra.
\item \textbf{Fakultná guľovačka} -- v prípade dobrých snehových podmienok organizovaná guľovačka. Pravdepodobne v Lužánkách, pred FI nie je dostatok miesta (respektíve riziko zasiahnutia nezúčastnených je príliš veľké). Predbežný termín: v prípade dobrých snehových podmienok počas skúškového.
\item Samozrejmosťou je pokračovanie zabehnutých nosných aktivít: InterLoS, InterSoB, KSI, K-SCUK a KSIlet.
\end{itemize}

\subsection*{Plány do blízkej budúcnosti}
\begin{itemize}[itemsep=0pt]
\item Zriadiť mailing list pre všetkých členov (aktívnych aj sympatizujúcich). Tento mailing list bude fungovať ako primárny informačný kanál o organizovaných aktivitách.
\item Zriadiť mailing list na členov výboru. Bude fungovať ako kontaktná adresa pre vedenie FI a interný komunikačný kanál výboru.
\item Dohodnúť s tajomníčkou FI, aby bol kľúč a miestnosť A220 dostupná nie len InterLoSu, ale spolku ako celku, teda organizátorom všetkých zúčastnených aktivít (InterLoS, InterSoB, K-SCUK, KSI, KSIlet). Ideálne bez obmedzenia na dennú dobu. Samozrejme rešpektujeme, že akékolvek zasadnutia zamestnancov, semináre a pod.\ majú prioritu a učebňa im bude na počkanie uvoľnená. Oprávnené osoby na prevzatie kľúča bude minimálne výbor spolku, ideálne by bolo, aby výbor dokázal oprávnené osoby pridávať/odoberať bez konzulácie s vedením FI.
\item Založit jednoduché webové stránky spolku, zverejniť stanovy, zápis zo zakladacej schôdze a zoznam členov.
\item Spolok je treba zapísať do zoznamu spolkov MU. So zápisom nám pomôže MUNIE, potrebný je názov a kontaktná osoba, môže byť uvedené aj sídlo, webstránky/Facebook, logo a krátky popis činnosti.
\item Pridať všetkých nových členov spolku do existujúcej skupiny na Facebooku.
\end{itemize}

\subsection*{Rôzne}
\begin{itemize}[itemsep=0pt]
\item V stredu 10.\ 12.\ 2014 večer sa uskutoční zasadnutie snemu Masarykovy studentské unie (MUNIE). Členmi je mnoho spolkov MU. Treba zvážiť, či sa tiež chceme stať členom.
\item SU FI plánuje opäť sa viac zapojiť do diania na FI. Plánovaný je ples, hravé odpoledne, prípadne obnovenie premietania filmov v posluchárňach D. Spolupráca so spolkom je obojstranne vítaná.
\item Myšlienka: Presunom študijného oddelenia sa vianočný stromček dostal na exponovanejšie miesto. Dalo by sa pod neho umiestniť \uv{dary} pre prednášajúcich FI (ľahko recesistické).
\item Druhý sponzorský dar spolku boli 2 hrnčeky s logom Red Hat. Nechávame ich v kuchynke na 2. poschodí A, vedľa perspektívnej tímovkovej miestnosti A220.
\end{itemize}
\textbf{Zapísal:} Martin Ukrop\\
\textbf{Skontroloval:} Jan Drábek

\end{document}
