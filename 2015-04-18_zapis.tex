\documentclass[11pt,a4paper]{article}

% ===== LOADING PACKAGES =====
% language settings, main documnet language last
\usepackage[czech]{babel}
% enabling new fonts support (nicer)
\usepackage{lmodern}
% setting input encoding
\usepackage[utf8]{inputenc}
% setting output encoding
\usepackage[T1]{fontenc}
% set page margins
\usepackage[top=2cm, bottom=2cm, left=2cm, right=2cm]{geometry}
% package to make bullet list nicer
\usepackage{enumitem}
% setting custom colors for links
\usepackage{xcolor}
\definecolor{dark-red}{rgb}{0.6,0.15,0.15}
\definecolor{dark-green}{rgb}{0.15,0.4,0.15}
\definecolor{medium-blue}{rgb}{0,0,0.5}
% generating hyperlinks in document
\usepackage{url}
\usepackage[plainpages=false, 	% get the page numbering correctly
            pdfpagelabels, 		% write arabic labels to all pages
            unicode,	 			% allow unicode characters in links
            colorlinks=true, 	% use colored links instead of boxed
            linkcolor={dark-red},
            citecolor={dark-green},
            urlcolor={medium-blue}
			]{hyperref}

\begin{document}
\section*{Zápis ze schůze Spolku přátel severské zvěře}
\textbf{Datum:} 18.~4.~2015\\
\textbf{Přítomní:} Martin Hanžl, Martina Krasnayová, Jan Mrázek, Vladimír Sedláček, Karel Kubíček, Tomáš Effenberger, dále přes internet Jan Drábek a po telefonu Martin Ukrop\\
\textbf{Místo schůze:} Rozhledna Babí lom \\
\subsection*{Průběh schůze}
Zrovna se konal Ksílet, akce pod záštitou Korespondenčního semináře KSI, který je pod záštitou Spolku přátel severské zvěře a slouží mimo jiné k propagaci Fakulty informatiky Masarykovy univerzity a Masarykovy univerzity obecně. Kromě několika dalších členů Spolku přátel severské zvěře se Ksíletu účastnili i členové výboru Spolku přátel severské zvěře, a to jmenovitě Martin Hanžl, Martina Krasanyová a Jan Mrázek a dále budoucí studentka Fakulty informatiky Masarykovy univerzity Dominika Krejčí. Protože se již nyní významným dílem podílí na propagaci Fakulty informatiky Masarykovy univerzity a Masarykovy univerzity obecně, rozhodli jsme se ji přijmout mezi členy Spolku přátel severské zvěře. Přijímání nových členů Spolku přátel severské zvěře je ovšem možné pouze na schůzích Spolku přátel severské zvěře, o nichž je informován výbor Spolku přátel severské zvěře, museli jsme tedy nejdříve uspořádat schůzi. Toto nám umožnil rychlý telefonický kontakt Martina Ukropa a až do Norska internetový kontakt Jana Drábka, kteří s konáním schůze a následným přijetím Dominiky Krejčí do Spolku přátel severské zvěře souhlasili. Tito dva členové výboru Spolku přátel severské zvěře se tedy schůze přímo nezúčastnili, ale protože na schůzi byli přítomni tři z pěti členů výboru Spolku přátel severské zvěře, mohli jsme provést přijímací rituál. Jali jsme se stoupat schody až na vrchol rozhledny, kde jsme za dozoru několika dalších členů Spolku přátel severské zvěře a účastníků Ksíletu Dominiku Krejčí přijali mezi členy Spolku přátel severské zvěře. V nastálé euforii jsme ještě mezi členy Spolku přátel severské zvěře přijali také Adélu Miklíkovou, Evu Šmijákovou a Tomáše Jadrného. O těchto třech však nebyl informován zbytek výboru Spolku přátel severské zvěře a jejich přijetí bude na programu další schůze Spolku přátel severské zvěře. Poté byla schůze rozpuštěna a vydali jsme se dolů z rozhledny, protože tam strašně foukalo.\\
\\
\textbf{Sepsal:} Martin Hanžl\\
\textbf{Zkontrolovali:} Martina Krasnayová, Jan Mrázek, Dominika Krejčí
\end{document}